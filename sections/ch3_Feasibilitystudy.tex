\chapter{Requirement Analysis}\label{ch:req_analysis}
\section{Functional Requirements}
Functional requirements are detailed descriptions of the features, capabilities, and interactions that a software system or application must possess in order to fulfill its intended purpose. The functional requirements for our project are given below :
\begin{itemize}
    \item The app should provide the ability to capture images of physical medical reports using the device's camera.
    \item AI algorithms should be able to categorize medical reports based on patient identifiers, report types, and medical procedures.
    \item The app should be able to automatically organize and tag reports for efficient storage and retrieval.
    \item Users should be able to share reports through QR code.
    \item Users should be able to scan codes to access the reports.
    \item The AI module should identify anomalies or unusual patterns in reports to verify the report.
    \item Users should be able to send reports to other user.
\end{itemize}

\section{Non-Functional Requirement}
Non-functional requirement are a set of specifications that describe the system's operation capabilities and constraints and attempt to improve its functionality.
The non-functional requirements for our project are given below :
\begin{itemize}
    \item The application should be user-friendly and intuitive for ease of use.
    \item The application should be scalable and should accommodate a growing user base.
    \item The application should be efficient with better response time
    \item The application should have user-friendly UI/UX.
    \item The project should be tamper proof and secure digital report storage
    \item The system should be able to handle all types of medical reports.
\end{itemize}


\section{Feasibility Study}
\subsection{Eonomic Feasibility}
The proposed Medical Report Scanning and Organization App demonstrates strong economic viability. Initial costs includes development, integration, training, and quality assurance. Revenue streams, including subscription models and data analytics services, are anticipated to provide significant returns. The app's potential to save time, enhance efficiency, and improve patient care contributes to a positive cost-benefit analysis. With scalability and innovation potential, the app is sure for sustained success within the dynamic digital healthcare landscape.
\subsection{Operational Feasibility}
 Seeing how well the app works in real-life situations, it's clear that it's a practical and user-friendly solution. The app's easy-to-use design makes it simple for doctors, patients to use without much trouble. It can easily fit in with the technology and systems that healthcare places already have. This means it won't cause a lot of disruption and can be added to daily routines without much problem. The app is built in a way that allows it to grow as more people use it, and it can help doctors and healthcare teams work together better by making it easier to access and share important medical reports. Overall, the app fits well into how things already work in healthcare and can help make processes smoother and more efficient.
 \subsection{Technical Feasibility}
Utilizing React Native to construct the Medical Report Scanning and Organization App is both practical and advantageous. React Native's compatibility across devices and platforms ensures accessibility, while its integration with Optical Character Recognition (OCR) and Artificial Intelligence (AI) technologies is well-supported and documented. The app's adaptability to evolving technologies, modular structure, and accessible development tools further reinforce its technical feasibility, making it a sound choice for creating a user-friendly and future-ready solution for medical report management.