\chapter{Introduction}\label{ch:overview}
\section{Background}
Optical Character Recognition (OCR) technology is a process by which printed or handwritten text is converted into machine-readable text. It involves using computer algorithms and pattern recognition to analyze images of text and extract the characters and words from them. OCR technology is widely used for digitizing physical documents, enabling text search and analysis in digital archives, and facilitating data entry automation.

\noindent
Digitalizing reports using OCR involves the application of Optical Character Recognition technology to convert printed or handwritten textual content within reports into machine-readable data. OCR algorithms employ image preprocessing techniques to enhance the quality of scanned documents, followed by pattern recognition methods to extract individual characters and words. This converted text is then digitized and encoded into a format that can be stored, edited, and analyzed digitally.The combination of OCR and AI technologies in the app brings a major change to how medical reports are managed. AI works alongside OCR to smartly sort and understand the digital content. It automatically organizes reports by patient names, report types, and medical details, making it faster for healthcare providers to work with documents. This helps to make better decisions and provide personalized care.


\section{Motivation}
In Nepal's changing healthcare world, where technology is getting better to help patients, there's a big problem we need to solve: how to handle medical papers well.
Lack of proper data and management system has become a hassle for the health consumer.
There's lots of patient information, and we need to find it quickly. We need new ideas to make things easier. That's where the Mero Health App comes in. It uses smart tools like OCR and AI to make medical papers easy to use. This app can really change how we deal with medical reports in Nepal and make things better for everyone. When we look at Nepal, bringing OCR and AI together in the app can bring big changes. OCR changes paper papers into digital words, so we can find things fast, even if we're far apart. AI helps put papers in groups by patient names, types of reports, and medical stuff. This makes dealing with papers easy, helps doctors make good choices, and focuses on patients. This mix not only makes healthcare work better but also lines up with Nepal's goal of good and safe healthcare for everyone.

\section{Objective}
\begin{itemize}
    \item To streamline medical document handling by digitizing and organizing paper-based reports using OCR and AI technologies.
    \item To promote collaborative care by facilitating sharing of reports among healthcare providers.
\end{itemize}

\section{Scope of the Work}
The project is about creating a smart mobile app that makes managing medical reports much easier. This app is carefully designed to make the process of handling medical papers simple and efficient. It does this by using special technologies called Optical Character Recognition (OCR) and Artificial Intelligence (AI). The main goal of the app is to solve the problems that come with dealing with traditional paper-based medical documents. This app helps both doctors and patients by providing a secure, easy-to-use platform. This platform can turn paper reports into digital ones, organize them, and help find important medical information quickly.

To make this happen, the app has a user-friendly design. People can easily take pictures of paper medical reports using their phone cameras. The app uses OCR to turn these pictures into words that can be easily searched and edited. AI also helps by organizing the reports automatically based on things like patient names and the type of report. This makes it much quicker to find the right information, helping doctors make better decisions.

The app takes data security seriously. It keeps sensitive patient information safe using strong encryption and secure cloud storage. This means that patient data is kept private, but doctors and patients can still access it whenever they need it. The combination of OCR and AI not only improves how medical reports are managed but also supports Nepal's goal of better healthcare. This app has the potential to change how medical information is handled, which could lead to better healthcare outcomes for people in Nepal.
\section{Application}
Our system project has significant potential for various major applications within the healthcare sector. These include:
\subsection*{Healthcare Professionals and Clinics} 
For healthcare professionals and clinics, the app presents an opportunity for enhanced diagnostics, streamlining the retrieval of past medical reports to inform accurate diagnoses and well-informed treatment strategies. During patient consultations, physicians can readily access organized medical records, facilitating more insightful discussions and informed decision-making. Additionally, the app supports efficient treatment monitoring by enabling healthcare providers to seamlessly track patient progress over time through a comparison of historical and recent medical reports.
\subsection*{Patients and Personal Health Management}
Patients themselves stand to benefit significantly from the app's capabilities. By maintaining personalized digital health records, individuals can take charge of their health management journey, incorporating lab results, prescriptions, and imaging reports for informed decision-making and self-care. This empowerment extends to engagement with healthcare providers, as patients can actively participate by conveniently sharing digital reports with specialists and practitioners.
\subsection*{Healthcare Administration and Records Management}
Administrative aspects of healthcare management can also be streamlined. The transition to digital records facilitates simplified records keeping within healthcare facilities, reducing the need for physical storage and enhancing administrative efficiency. Furthermore, the app aids insurance processing by providing insurers with secure access to digitized medical reports, expediting claims processing and verification.
\subsection*{Telemedicine and Remote Consultations}
In the realm of telemedicine and remote consultations, the app becomes a valuable tool for remote diagnoses. Telemedicine practitioners can remotely access digitized reports to provide accurate assessments and appropriate treatment recommendations, bridging geographical gaps in healthcare access. This feature is especially pertinent in facilitating efficient virtual care, enabling patients to transmit their medical history to virtual healthcare providers for comprehensive online consultations.
\subsection*{Health and Wellness Apps Integration}
Integration with health and wellness apps further amplifies the app's utility. It offers users a comprehensive view of their health status, facilitating personalized fitness and wellness plans that are grounded in a holistic understanding of their medical history.
